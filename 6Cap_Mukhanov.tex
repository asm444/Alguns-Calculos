\documentclass[a4paper,12pt]{article}
\usepackage[top=2cm, bottom=2cm, left=2.5cm, right=2.5cm]{geometry}
\usepackage[utf8]{inputenc}
\usepackage{amsmath, amsfonts, amssymb}
\usepackage{float}
\usepackage{graphicx}
\usepackage{verbatim}

\author{Arthur de Souza Molina}
\title{Resolução de Alguns Cálculos do capitulo , Mukhanov}

\DeclareUnicodeCharacter{2212}{-}
\begin{document}
	Arthur de Souza Molina
	
	Resolução de Alguns Cálculos:
	
	Mukhanov capitulo 6: Instabilidade Gravitacional na teoria Newtoniana
	
\begin{center}
	\textbf{6.1 Equações básicas}

\end{center}

Em grandes escalas, a matéria pode ser considerada um fluido perfeito, e sua distribuição de energia pode ser caracterizada por $\varepsilon (\textbf{x},t)$, a entropia por unidade de massa $S(\textbf{x},t)$, e o vetor de velocidade $\textbf{V}(\textbf{x},t)$.

Se considerarmos um volume fixo não móvel, a sua variação de massa pode ser descrita como

\begin{equation}
		\dfrac{dM}{dt} = \int_{\Delta V} \dfrac{\partial \varepsilon (x,t)}{\partial t}
\end{equation}

A variação de Massa também pode ser descrita pelo fluxo de matéria pela superfície ao redor do volume fixo

\begin{equation}
	\dfrac{dM}{dt} = - \oint \varepsilon \textbf{V} d\sigma = - \int_{\Delta V} \nabla \,(\varepsilon \, \textbf{V}) dV
\end{equation}

Essas relações somente são consistentes se

\begin{equation}
	\dfrac{\partial \varepsilon}{\partial t} +\nabla \,(\varepsilon \, \textbf{V}) = 0
\end{equation}

A aceleração gravitacional $\textbf{g}$ em um pequeno elemento com massa é determinada pela força gravitacional

\begin{equation}
	\textbf{F}_{gr} = - \Delta M \cdot \nabla \phi
\end{equation}

onde o potencial gravitacional é dado por $\phi$ e pela pressão $\textit{p}$
\begin{equation}
	\textbf{F}_{pr} = - \oint p \cdot\ d\sigma = - \int_{\Delta V} \nabla p\ d\sigma \approx - \nabla p \ \Delta V
\end{equation}

com 
\begin{equation}
	\textbf{g} \equiv \dfrac{d \textbf{V} (\textbf{x} (t), t)}{dt} = \left( \dfrac{\partial \textbf{V}}{\partial t} \right)_x + \dfrac{dx^i(t)}{dt}\left( \dfrac{\partial \textbf{V}}{\partial x^i} \right) = \frac{\partial \textbf{V}}{\partial t} + (\textbf{V} \cdot \nabla ) \textbf{V}
\end{equation}
\newline
pela lei da força de Newton (Segunda Lei de Newton)

\begin{equation}
	\Delta M \cdot \textbf{g} = \textbf{F}_{gr} + \textbf{F}_{pr}
\end{equation}
\newline
$$\Delta M \cdot \textbf{g} -  \textbf{F}_{gr} - \textbf{F}_{pr} = 0 $$

$$\Delta M \cdot \textbf{g}  + \nabla p \ \Delta V + \Delta M \cdot \nabla \phi=0 $$

com $\Delta M \equiv \Delta V \cdot \varepsilon$, e portanto $\Delta V = \dfrac{\Delta M}{\varepsilon}$
$$\Delta M \cdot \textbf{g}  + \nabla p \,\dfrac{\Delta M}{\varepsilon} + \Delta M \cdot \nabla \phi=0 $$

$$\textbf{g}  + \dfrac{\nabla p}{\varepsilon} + \nabla \phi=0 $$
ao substituir a equação (6), torna-se a equação de Euler

\begin{equation}
	\dfrac{\partial \textbf{V} }{\partial t} + (\textbf{V} \cdot \nabla ) \textbf{V} + \frac{\nabla p}{\varepsilon} + \nabla \phi =0
\end{equation}

A conservação de entropia negligência a dissipação de energia, logo a entropia para um pequeno elemento de matéria é conservada:

\begin{equation}
	\dfrac{dS(\textbf{x},t)}{dt} = \dfrac{\partial S}{\partial t} + ( \textbf{V} \cdot \nabla) S = 0
\end{equation}

A equação que determina o potencial gravitacional é conhecida como equação de Poisson

\begin{equation}
	\Delta\phi = 4\pi G\varepsilon
\end{equation}

Com as equações $(3), (6)\to (10)$ tomadas em conjunto com a equação de estado

\begin{equation}
	p= p(\varepsilon , S)
\end{equation}

formam um conjunto completo de sete equações que, em princípio, nos permite determinar
as sete funções desconhecidas $\varepsilon, \textbf{V}, S, \phi, p$. As equações hidrodinâmicas  não são lineares, e em geral não é fácil encontrar suas soluções. No entanto, para estudar o comportamento de pequenas perturbações em torno de um fundo homogêneo e isotrópico, é apropriado torna-las lineares.

\begin{center}	
	\textbf{6.2 Teoria Jeans}
\end{center}

Vamos primeiro considerar um universo não expansível estático, assumindo o homogêneo,
fundo isotrópico com densidade de matéria independente do tempo constante: $\varepsilon (x,t)= const$. Na verdade, a densidade de energia permanece inalterada apenas se a matéria estiver em repouso e a força gravitacional $F \propto\nabla\phi$, que desaparece. Logo, a equação de Poisson não é satisfeita, essa inconsistência pode, em principio, se considerarmos um universo estático, onde a força gravitacional da matéria é compensada pela constante cosmológica apropriada escolhida.

Ao perturbar a distribuição de matéria, temos

\begin{equation}
	\varepsilon (\textbf{x},t) = \varepsilon_0 + \delta\varepsilon (\textbf{x},t),\,\, \textbf{V} (\textbf{x},t) = \textbf{V}_0 +\delta\textbf{V} (\textbf{x},t) 
\end{equation}

$$\phi (\textbf{x},t) = \phi_0 + \delta\phi (\textbf{x},t), S (\textbf{x},t)= S_0 + \delta S (\textbf{x},t)$$

onde cada variação $\delta\varepsilon \ll \varepsilon_0$, etc..
\newline
A pressão é dada por

\begin{equation}
	p (\textbf{x},t) = p( \varepsilon_0 + \delta\varepsilon (\textbf{x},t), S_0 + \delta S (\textbf{x},t) ) = p_0 +\delta p (\textbf{x},t) 
\end{equation}

$\delta p$ pode ser escrito em termos da densidade de energia e pertubações de entropia como

\begin{equation}
	\delta p = c_s^2\delta\varepsilon + \sigma\delta S
\end{equation}

onde $c^2_s \equiv \left(\dfrac{\partial p}{\partial\varepsilon}\right)_s$ é o quadrado da velocidade do som e $\sigma \equiv \left(\dfrac{\partial p}{\partial S}\right)_\varepsilon$. Para a matéria não relativística, a velocidade do som é muito menor que a velocidade da luz.

Substituindo (12) e (14) em (3), (8) - (10) e mantendo apenas os termos
que são lineares nas perturbações, obtemos:
\newline
$$\dfrac{\partial \varepsilon}{\partial t} +\nabla \,(\varepsilon \, \textbf{V}) = 0$$

\begin{equation}
	\dfrac{\partial\delta\varepsilon}{\partial t} + \varepsilon_0 \nabla (\delta\textbf{v}) = 0
\end{equation}
\newline

$$\dfrac{\partial\textbf{V}}{\partial t} + (\textbf{V} \cdot \nabla ) \textbf{V} + \frac{\nabla p}{\varepsilon} + \nabla \phi =0$$

$$\dfrac{\partial\delta\textbf{v}}{\partial t} + (\delta\textbf{v} \cdot \nabla ) \delta\textbf{v} + \frac{\nabla \delta p}{\varepsilon_0} + \nabla \delta\phi =0$$

$$\dfrac{\partial\delta\textbf{v}}{\partial t} + \frac{\nabla (c_s^2\delta\varepsilon + \sigma\delta S )}{\varepsilon_0} + \nabla \delta\phi =0$$

$$\dfrac{\partial\delta\textbf{v}}{\partial t} +\frac{\nabla c_s^2\delta\varepsilon }{\varepsilon_0} + \frac{\nabla \sigma\delta S }{\varepsilon_0} + \nabla \delta\phi =0$$

\begin{equation}
	\dfrac{\partial\delta\textbf{v}}{\partial t} + \dfrac{c^2_s}{\varepsilon_0} \nabla\delta\varepsilon + \dfrac{\sigma}{\varepsilon_0}\nabla\delta S + \nabla\delta\phi = 0
\end{equation}
\newline

$$\dfrac{\partial S}{\partial t} = 0$$

\begin{equation}
	\dfrac{\partial\delta S}{\partial t} = 0
\end{equation}
\newline
$$\Delta\phi = 4\pi G\varepsilon$$

\begin{equation}
	\Delta\delta\phi = 4\pi G\delta\varepsilon
\end{equation}

A equação (17) possui solução simples, dado que
\begin{equation}
	\delta S (\textbf{x},t) =\delta S (\textbf{x}),
\end{equation}

isso indica que a entropia é invariante em relação ao tempo, mas varia em coordenadas arbitrárias.

Tomando a divergente da equação (16) e escrevendo $\nabla\delta S$ e $\Delta\delta\phi$ em termos de $\delta\varepsilon$, temos

$$\dfrac{\partial\delta\textbf{v}}{\partial t} + \dfrac{c^2_s}{\varepsilon_0} \nabla\delta\varepsilon + \dfrac{\sigma}{\varepsilon_0}\nabla\delta S + \nabla\delta\phi = 0$$

$$\nabla\cdot \left( \dfrac{\partial\delta\textbf{v}}{\partial t} + \dfrac{c^2_s}{\varepsilon_0} \nabla\delta\varepsilon + \dfrac{\sigma}{\varepsilon_0}\nabla\delta S + \nabla\delta\phi \right)= \nabla\cdot 0$$

$$\nabla\cdot \left( \dfrac{\partial\delta\textbf{v}}{\partial t} \right) + \nabla\cdot \left( \dfrac{c^2_s}{\varepsilon_0} \nabla\delta\varepsilon \right) +\nabla\cdot \left( \dfrac{\sigma}{\varepsilon_0}\nabla\delta S \right) + \nabla\cdot \left( \nabla\delta\phi \right) = 0$$

$$\dfrac{\partial\Delta\textbf{v}}{\partial t} + \dfrac{c^2_s}{\varepsilon_0} \Delta\delta\varepsilon  + \dfrac{\sigma}{\delta\varepsilon_0}\Delta\delta S (\textbf{x})+  \Delta\delta\phi = 0$$

dado que $\dfrac{\partial\delta\varepsilon}{\partial t} + \varepsilon_0 \nabla (\delta\textbf{v}) = 0$ e $\dfrac{\partial\Delta\textbf{v}}{\partial t}= - \dfrac{\partial^2\delta\varepsilon}{\varepsilon_0\partial t^2} $ então

$$- \dfrac{\partial^2\delta\varepsilon}{\varepsilon_0\partial t^2}+\dfrac{c^2_s}{\varepsilon_0} \Delta\delta\varepsilon  + \dfrac{\sigma}{\varepsilon_0}\Delta\delta S (\textbf{x})+  \Delta\delta\phi = 0$$

$$\dfrac{\partial^2\delta\varepsilon}{\partial t^2}-c^2_s \Delta\delta\varepsilon  -\sigma\Delta\delta S (\textbf{x})-  \varepsilon_0\Delta\delta\phi = 0$$

$$\dfrac{\partial^2\delta\varepsilon}{\partial t^2}-c^2_s \Delta\delta\varepsilon  -\sigma\Delta\delta S (\textbf{x})-  4\pi G\varepsilon_0\delta\varepsilon = 0$$

\begin{equation}
	\dfrac{\partial^2\delta\varepsilon}{\partial t^2} - c^2_s\Delta\delta\varepsilon - 4\pi G\varepsilon_0\delta\varepsilon = \sigma\Delta\delta S(\textbf{x})
\end{equation}

Esta é uma equação linear fechada para $\delta\varepsilon$, onde a perturbação de entropia serve como um
determinada fonte.

\begin{center}
	\textbf{6.2.1 Pertubações Adiabáticas}
\end{center}
Primeiro, podemos supor as pertubações de entropia ausentes, isto é, $\delta S=0$. O
coeficientes em (20) não dependem das coordenadas espaciais, portanto, ao tomar o
Transformada de Fourier,

\begin{equation}
	\delta\varepsilon (\textbf{x} , t ) = \int \delta\varepsilon_k (t)  \dfrac{e^{ik\textbf{x}}d^3 k}{(2\pi)^{2/3}}
\end{equation}

obtemos um conjunto de equações diferenciais ordinárias independentes para as
Coeficientes de Fourier $\delta\varepsilon_k (t)$

$$\dfrac{\partial^2\delta\varepsilon}{\partial t^2} - c^2_s\Delta\delta\varepsilon - 4\pi G\varepsilon_0\delta\varepsilon = 0$$

$$\dfrac{\partial^2 \left(\int \delta\varepsilon_k (t)  \dfrac{e^{ik\textbf{x}}d^3 k}{(2\pi)^{2/3}} \right)}{\partial t^2} - c^2_s\Delta\left(\int \delta\varepsilon_k (t)  \dfrac{e^{ik\textbf{x}}d^3 k}{(2\pi)^{2/3}} \right) - 4\pi G\varepsilon_0\left(\int \delta\varepsilon_k (t)  \dfrac{e^{ik\textbf{x}}d^3 k}{(2\pi)^{2/3}} \right) = 0$$

$$ \dfrac{\partial^2 }{\partial t^2}\left(\int \delta\varepsilon_k (t)  \dfrac{e^{ik\textbf{x}}d^3 k}{(2\pi)^{2/3}} \right) - c^2_s\Delta\left(\int \delta\varepsilon_k (t) \dfrac{e^{ik\textbf{x}} d^3 k}{(2\pi)^{2/3}} \right) - 4\pi G\varepsilon_0\delta\varepsilon_k (t)\left(\int   \dfrac{e^{ik\textbf{x}}d^3 k}{(2\pi)^{2/3}} \right) = 0$$

dado que $\Delta = \nabla^2 = \dfrac{\partial^2}{\partial\textbf{x}^2} = \dfrac{\partial^2}{\partial x^2} + \dfrac{\partial^2}{\partial y^2} + \dfrac{\partial^2}{\partial z^2}$, logo

$$\dfrac{\partial^2 }{\partial t^2}\left(\int \delta\varepsilon_k (t)  \dfrac{e^{ik\textbf{x}}d^3 k}{(2\pi)^{2/3}} \right)  - c^2_s\left(\int  \dfrac{\partial^2  (\delta\varepsilon_k (t))}{\partial\textbf{x}^2} \dfrac{e^{ik\textbf{x}}d^3 k}{(2\pi)^{2/3}} \right) - 4\pi G\varepsilon_0  \left(\int \delta\varepsilon_k (t)  \dfrac{e^{ik\textbf{x}}d^3 k}{(2\pi)^{2/3}} \right) = 0$$

pela propriedade da transformada de Fourier de derivadas, temos 
$$\int  \dfrac{\partial^2  (\delta\varepsilon_k (t))}{\partial\textbf{x}^2} \dfrac{e^{ik\textbf{x}}d^3 k}{(2\pi)^{2/3}} = (|k|i)^2\int  \delta\varepsilon_k (t) \dfrac{e^{ik\textbf{x}}d^3 k}{(2\pi)^{2/3}}$$

portanto
$$\dfrac{\partial^2 }{\partial t^2}\left(\int \delta\varepsilon_k (t)  \dfrac{e^{ik\textbf{x}}d^3 k}{(2\pi)^{2/3}} \right)  - c^2_s \left((ki)^2\int  \delta\varepsilon_k (t) \dfrac{e^{ik\textbf{x}}d^3 k}{(2\pi)^{2/3}}\right) - 4\pi G\varepsilon_0  \left(\int \delta\varepsilon_k (t)  \dfrac{e^{ik\textbf{x}}d^3 k}{(2\pi)^{2/3}} \right) = 0$$

$$\dfrac{\partial^2 }{\partial t^2}\left(\int \delta\varepsilon_k (t)  \dfrac{e^{ik\textbf{x}}d^3 k}{(2\pi)^{2/3}} \right)  + |k|^2c^2_s\int  \delta\varepsilon_k (t) \dfrac{e^{ik\textbf{x}}d^3 k}{(2\pi)^{2/3}} - 4\pi G\varepsilon_0  \left(\int \delta\varepsilon_k (t)  \dfrac{e^{ik\textbf{x}}d^3 k}{(2\pi)^{2/3}} \right) = 0$$

$$\dfrac{\partial^2 }{\partial t^2} \left(\int \delta\varepsilon_k (t)  \dfrac{e^{ik\textbf{x}}d^3 k}{(2\pi)^{2/3}} \right) + (|k|^2c^2_s - 4\pi G\varepsilon_0)  \left(\int \delta\varepsilon_k (t)  \dfrac{e^{ik\textbf{x}}d^3 k}{(2\pi)^{2/3}} \right) = 0$$

$$ \left(\int \dfrac{\partial^2 }{\partial t^2}( \delta\varepsilon_k (t))  \dfrac{e^{ik\textbf{x}}d^3 k}{(2\pi)^{2/3}} \right) + \int (|k|^2c^2_s - 4\pi G\varepsilon_0)\delta\varepsilon_k (t) \dfrac{ e^{ik\textbf{x}}d^3 k}{(2\pi)^{2/3}} = 0$$

$$ \int \left(\dfrac{\partial^2 (\delta\varepsilon_k (t))}{\partial t^2} +(|k|^2c^2_s - 4\pi G\varepsilon_0)\delta\varepsilon_k (t)\right)  \dfrac{e^{ik\textbf{x}}d^3 k}{(2\pi)^{2/3}} = 0$$

multiplicando por $e^{i k^\prime \textbf{x}}$ em ambos os lados, temos
$$ e^{i k^\prime \textbf{x}} \cdot \int \left(\dfrac{\partial^2 (\delta\varepsilon_k (t))}{\partial t^2} +(|k|^2c^2_s - 4\pi G\varepsilon_0)\delta\varepsilon_k (t)\right)\dfrac{e^{ik\textbf{x}}d^3 k}{(2\pi)^{2/3}} = 0 \cdot  e^{i k^\prime \textbf{x}}$$

$$ \int \left(\dfrac{\partial^2 (\delta\varepsilon_k (t))}{\partial t^2} +(|k|^2c^2_s - 4\pi G\varepsilon_0)\delta\varepsilon_k (t)\right) e^{i k^\prime \textbf{x}}  \dfrac{e^{ik\textbf{x}}d^3 k}{(2\pi)^{2/3}} = 0$$

$$ \int \cdot \left(\int \left(\dfrac{\partial^2 (\delta\varepsilon_k (t))}{\partial t^2} +(|k|^2c^2_s - 4\pi G\varepsilon_0)\delta\varepsilon_k (t)\right) e^{i k^\prime \textbf{x}}  \dfrac{e^{ik\textbf{x}}d^3 k}{(2\pi)^{2/3}}\right) d\textbf{x} =\int \left( 0 \right) d\textbf{x}$$

$$  \int \left(\dfrac{\partial^2 (\delta\varepsilon_k (t))}{\partial t^2} +(|k|^2c^2_s - 4\pi G\varepsilon_0)\delta\varepsilon_k (t)\right)\cdot \left(\int \dfrac{e^{ i (k^\prime \textbf{x} +  k\textbf{x}) }}{(2\pi)^{2/3}}d\textbf{x}\right) d^3k   =0$$

$$ \int \left(\dfrac{\partial^2 (\delta\varepsilon_k (t))}{\partial t^2} +(|k|^2c^2_s - 4\pi G\varepsilon_0)\delta\varepsilon_k (t)\right) \cdot\left(\int  \dfrac{e^{ i \textbf{x}(k^\prime  + k) }}{(2\pi)^{2/3}}d\textbf{x} \right)d^3k  =0$$

$$  \int \left(\dfrac{\partial^2 (\delta\varepsilon_k (t))}{\partial t^2} +(|k|^2c^2_s - 4\pi G\varepsilon_0)\delta\varepsilon_k (t)\right) \delta (k^\prime + k) d^3k  =0$$

$$  \int \delta (k^\prime + k)\left(\dfrac{\partial^2 (\delta\varepsilon_k (t))}{\partial t^2} +(|k|^2c^2_s - 4\pi G\varepsilon_0)\delta\varepsilon_k (t)\right)  d^3k  =0$$

para a integral definida ser equivalente a zero, o integrando deve ser igual a zero, logo

\begin{equation}
	\dfrac{\partial^2 (\delta\varepsilon_k (t))}{\partial t^2} +(k^2c^2_s - 4\pi G\varepsilon_0)\delta\varepsilon_k (t) = 0
\end{equation}

sendo $k = |k|$ e uma possível solução seria $\delta\varepsilon_k=\lambda e^{\omega t}$, portanto

$$\dfrac{\partial^2 (\lambda e^{\omega t})}{\partial t^2} +(k^2c^2_s - 4\pi G\varepsilon_0)\lambda e^{\omega t} = 0$$

$$\lambda\omega^2 e^{\omega t} +(k^2c^2_s - 4\pi G\varepsilon_0)\lambda e^{\omega t} = 0$$

$$\omega^2  + k^2c^2_s - 4\pi G\varepsilon_0 = 0$$

$$\omega^2  =  + 4\pi G\varepsilon_0 -k^2c^2_s $$

$$\omega = \pm \sqrt{4\pi G\varepsilon_0 -k^2c^2_s} =  \pm \sqrt{(-1)(k^2c^2_s - 4\pi G\varepsilon_0) }= \pm i \sqrt{k^2c^2_s - 4\pi G\varepsilon_0} $$

A equação (22) tem a solução
\begin{equation}
	\delta\varepsilon_k (t) \propto e^{(\pm \omega (t) t)}
\end{equation}

onde $\omega (t) = \sqrt{k^2c^2_s - 4\pi G \varepsilon_0}$. O comportamento da pertubação adiabática depende exclusivamente pelo sinal do expoente. 

Definindo o comprimento Jeans como
\begin{equation}
	\lambda_J = \dfrac{2\pi}{k_J} = c_S \left(\dfrac{\pi}{G\varepsilon_0} \right)^{1/2}
\end{equation}

de modo que $\omega (k_J) = 0$, concluímos que se $\lambda < \lambda_J$, as soluções descrevem as ondas sonoras

\begin{equation}
	\delta\varepsilon_k \propto \sin (\omega t + \mathbf{k}\mathbf{x} + \alpha)
\end{equation}

propagando com velocidade de fase
\begin{equation}
 	c_{fase} = \dfrac{\omega}{k}= c_s\sqrt{1 - \dfrac{k^2_J}{k}}.
\end{equation}

No limite $k \geq k_J$ ou em escalas muito pequenas ($\lambda \leq \lambda_J$) onde a gravidade é insignificante comparado com a pressão, temos $c_{fase} \to c_s$.
Em largas escalas a gravidade domina e $\lambda > \lambda_J$, temos
\begin{equation}
	\delta\varepsilon_k \propto e^{\pm |\omega| t }
\end{equation}

 Uma dessas soluções descrevem o comportamento exponencialmente rápido e não homogêneo, enquanto outras correspondem o modo decaimento. Onde $k \to 0, |\omega | t \to \dfrac{t}{t_{gr}}$, onde $t_{gr} \equiv (4\pi G\varepsilon_0)^{-1/2}$. Onde $t_{gr}$ é interpretado como o tempo característico de colapso para uma região com uma densidade de energia inicial $\varepsilon_0$.
 O comprimento Jeans $\lambda_J \sim c_s t_{gr} $ é o "comunicação de som" sobre a qual a pressão consegue reagir as mudanças da densidade de energia devido ao colapso gravitacional.
 
 Encontrando e analisando as equações para $\delta\mathbf{v}_k$ e $\delta\mathbf{\phi}_k$ para as ondas sonoras e para as pertubações em grandes escalas para o comprimento Jeans.  Problema (6.1)
 
 \begin{center}
 	\textbf{6.2.2 Pertubações de Vetor}
 \end{center}
 Substituindo $\delta\varepsilon = 0$ e $\delta S = 0$ na equação (20), obtemos uma solução não trivial para o sistema de equações hidrodinâmicas. Já para as equações (15)-(18) reduzimos em
 
 \begin{equation}
 	\nabla \delta\mathbf{v} = 0 ,\,\,\,\,\,\,\, \dfrac{\partial \mathbf{v}}{\partial t} = 0
 \end{equation}

Pela segunda equação, é possível notar que $\delta\mathbf{v}$ é uma função das coordenadas espaciais independente do tempo. E a primeira equação diz respeito a velocidade da pertubação de onda plana, isto é, $\delta\mathbf{v} = \mathbf{w_k} e^{i\mathbf{k}\mathbf{x}}$ e que a velocidade é perpendicular ao vetor de onda $\mathbf{k}$, ou seja,
\begin{equation}
	\mathbf{w_k} \cdot \mathbf{k} = 0
\end{equation}

 Essas perturbações vetoriais descrevem movimentos de cisalhamento do meio que não perturbam a densidade de energia. Porque existem duas direções perpendiculares independentes para $\mathbf{k}$, existem dois modos vetoriais independentes para um determinado $\mathbf{k}$.
  \begin{center}
 	\textbf{6.2.2 Pertubações de Entropia}
 \end{center}

Considerando que $\delta S \neq 0$, dada a transformação de Fourier da equação (20) e dado que $$\delta S = \int \delta S_k \dfrac{e^{ik\textbf{x}}d^3 k}{(2\pi)^{2/3}}, $$ obtendo uma solução semelhante a equação (22), temos

$$\dfrac{\partial^2\delta\varepsilon}{\partial t^2} - c^2_s\Delta\delta\varepsilon - 4\pi G\varepsilon_0\delta\varepsilon = \sigma\Delta \delta S$$

$\displaystyle\int \left(\dfrac{\partial^2 \delta\varepsilon_k (t)}{\partial t^2}  \dfrac{e^{ik\textbf{x}}}{(2\pi)^{2/3}} \right)d^3 k  - c^2_s\left( \int \delta\varepsilon_k (t)\Delta (e^{ik\textbf{x}})  \dfrac{d^3 k}{(2\pi)^{2/3}} \right) - 4\pi G\varepsilon_0\left(\int \delta\varepsilon_k (t)  \dfrac{e^{ik\textbf{x}}d^3 k}{(2\pi)^{2/3}} \right) = \sigma \int \left(\delta S_k \Delta (e^{ik\textbf{x}})\dfrac{}{(2\pi)^{2/3}} \right)d^3 k$

$\displaystyle\int \left(\dfrac{\partial^2 \delta\varepsilon_k (t)}{\partial t^2}  \dfrac{e^{ik\textbf{x}}}{(2\pi)^{2/3}} \right)d^3 k  - c^2_s\left( \int  \dfrac{\partial^2  (\delta\varepsilon_k (t))}{\partial\textbf{x}^2} \dfrac{e^{ik\textbf{x}}d^3 k}{(2\pi)^{2/3}} \right) - 4\pi G\varepsilon_0\left(\int \delta\varepsilon_k (t)  \dfrac{e^{ik\textbf{x}}d^3 k}{(2\pi)^{2/3}} \right) = \sigma  \left(\int  \dfrac{\partial^2  (\delta S_k)}{\partial\textbf{x}^2} \dfrac{e^{ik\textbf{x}}d^3 k}{(2\pi)^{2/3}} \right)$
\newline
pela propriedade da transformada de Fourier de derivadas, temos 

$$\int  \dfrac{\partial^2  (\delta\varepsilon_k (t))}{\partial\textbf{x}^2} \dfrac{e^{ik\textbf{x}}d^3 k}{(2\pi)^{2/3}} = (|k|i)^2\int  \delta\varepsilon_k (t) \dfrac{e^{ik\textbf{x}}d^3 k}{(2\pi)^{2/3}}$$
 e 
 
$$\int  \dfrac{\partial^2  (\delta S_k)}{\partial\textbf{x}^2} \dfrac{e^{ik\textbf{x}}d^3 k}{(2\pi)^{2/3}} = (|k|i)^2\int  \delta S_k  \dfrac{e^{ik\textbf{x}}d^3 k}{(2\pi)^{2/3}}$$

$\displaystyle\int \left(\dfrac{\partial^2 \delta\varepsilon_k (t)}{\partial t^2}  \dfrac{e^{ik\textbf{x}}}{(2\pi)^{2/3}} \right)d^3 k  - c^2_s (|k|i)^2\left(\int  \delta\varepsilon_k (t) \dfrac{e^{ik\textbf{x}}d^3 k}{(2\pi)^{2/3}} \right) - 4\pi G\varepsilon_0\left(\int \delta\varepsilon_k (t)  \dfrac{e^{ik\textbf{x}}d^3 k}{(2\pi)^{2/3}} \right) = \sigma (|k|i)^2 \left(\int  \delta S_k  \dfrac{e^{ik\textbf{x}}d^3 k}{(2\pi)^{2/3}} \right)$

$\displaystyle\int \left(\dfrac{\partial^2 \delta\varepsilon_k (t)}{\partial t^2}  \dfrac{e^{ik\textbf{x}}}{(2\pi)^{2/3}} \right)d^3 k  + c^2_s k^2\left(\int  \delta\varepsilon_k (t) \dfrac{e^{ik\textbf{x}}d^3 k}{(2\pi)^{2/3}} \right) - 4\pi G\varepsilon_0\left(\int \delta\varepsilon_k (t)  \dfrac{e^{ik\textbf{x}}d^3 k}{(2\pi)^{2/3}} \right) =- \sigma k^2 \left(\int  \delta S_k  \dfrac{e^{ik\textbf{x}}d^3 k}{(2\pi)^{2/3}} \right)$

$\displaystyle\int \left(\dfrac{\partial^2 \delta\varepsilon_k (t)}{\partial t^2}  \dfrac{e^{ik\textbf{x}}}{(2\pi)^{2/3}} \right)d^3 k  + c^2_s k^2\left(\int  \delta\varepsilon_k (t) \dfrac{e^{ik\textbf{x}}d^3 k}{(2\pi)^{2/3}} \right) - 4\pi G\varepsilon_0\left(\int \delta\varepsilon_k (t)  \dfrac{e^{ik\textbf{x}}d^3 k}{(2\pi)^{2/3}} \right) + \sigma k^2 \left(\int  \delta S_k  \dfrac{e^{ik\textbf{x}}d^3 k}{(2\pi)^{2/3}} \right)=0 $

$$\int \left(\left( \dfrac{\partial^2 \delta\varepsilon_k (t)}{\partial t^2}+ k^2c^2_s\delta\varepsilon_k (t)- 4\pi G\varepsilon_0\delta\varepsilon_k (t) + \sigma k^2\delta S_k\right)   \dfrac{e^{ik\textbf{x}}}{(2\pi)^{2/3}} \right)d^3 k =0 $$

multiplicando por $e^{i\mathbf{k}^\prime x}$

$$e^{i\mathbf{k}^\prime x}\left(\int \left(\left( \dfrac{\partial^2 \delta\varepsilon_k (t)}{\partial t^2}+ k^2c^2_s\delta\varepsilon_k (t)- 4\pi G\varepsilon_0\delta\varepsilon_k (t) + \sigma k^2\delta S_k\right)   \dfrac{e^{ik\textbf{x}}}{(2\pi)^{2/3}} \right)d^3 k \right) =0e^{i\mathbf{k}^\prime x} $$

$$\int \left(\left( \dfrac{\partial^2 \delta\varepsilon_k (t)}{\partial t^2}+ k^2c^2_s\delta\varepsilon_k (t)- 4\pi G\varepsilon_0\delta\varepsilon_k (t) + \sigma k^2\delta S_k\right)  e^{i\mathbf{k}^\prime x} \dfrac{e^{ik\textbf{x}}}{(2\pi)^{2/3}} \right)d^3 k  =0 $$

$$\int \left(\left( \dfrac{\partial^2 \delta\varepsilon_k (t)}{\partial t^2}+ k^2c^2_s\delta\varepsilon_k (t)- 4\pi G\varepsilon_0\delta\varepsilon_k (t) + \sigma k^2\delta S_k\right) \dfrac{e^{i(\mathbf{k}^\prime x+k\textbf{x})}}{(2\pi)^{2/3}} \right)d^3 k  =0 $$

$$\int \left(\left( \dfrac{\partial^2 \delta\varepsilon_k (t)}{\partial t^2}+ k^2c^2_s\delta\varepsilon_k (t)- 4\pi G\varepsilon_0\delta\varepsilon_k (t) + \sigma k^2\delta S_k\right) \dfrac{e^{ix(\mathbf{k}^\prime +k)}}{(2\pi)^{2/3}} \right)d^3 k  =0 $$

integrando os dois lados da igualdade em relação a x defida de $-\infty$ à $\infty$, temos

$$\int^{\infty}_{-\infty} \left( \int \left(\left( \dfrac{\partial^2 \delta\varepsilon_k (t)}{\partial t^2}+ k^2c^2_s\delta\varepsilon_k (t)- 4\pi G\varepsilon_0\delta\varepsilon_k (t) + \sigma k^2\delta S_k\right) \dfrac{e^{ix(\mathbf{k}^\prime +k)}}{(2\pi)^{2/3}} \right)d^3 k \right) dx  =\int^{\infty}_{-\infty}0dx $$

$$ \int \left(\left( \dfrac{\partial^2 \delta\varepsilon_k (t)}{\partial t^2}+ k^2c^2_s\delta\varepsilon_k (t)- 4\pi G\varepsilon_0\delta\varepsilon_k (t) + \sigma k^2\delta S_k\right) \int^{\infty}_{-\infty} \left(\dfrac{e^{ix(\mathbf{k}^\prime +k)}}{(2\pi)^{2/3}} \right)dx  \right)d^3 k  = 0 $$

pela definição do delta de Dirac,  obtemos
$$ \int \left(\left( \dfrac{\partial^2 \delta\varepsilon_k (t)}{\partial t^2}+ k^2c^2_s\delta\varepsilon_k (t)- 4\pi G\varepsilon_0\delta\varepsilon_k (t) + \sigma k^2\delta S_k\right) (\delta (k^\prime + k))  \right)d^3 k  = 0 $$

$$ \int \left(\delta (k^\prime + k)\left( \dfrac{\partial^2 \delta\varepsilon_k (t)}{\partial t^2}+ k^2c^2_s\delta\varepsilon_k (t)- 4\pi G\varepsilon_0\delta\varepsilon_k (t) + \sigma k^2\delta S_k\right)   \right)d^3 k  = 0 $$

dado que para a integral ser equivalente há zero, o integrando deverá ser igual há 0, logo
$$\dfrac{\partial^2 \delta\varepsilon_k (t)}{\partial t^2}+ k^2c^2_s\delta\varepsilon_k (t)- 4\pi G\varepsilon_0\delta\varepsilon_k (t) + \sigma k^2\delta S_k = 0 $$
\begin{equation}
	\dfrac{\partial^2 (\delta\varepsilon_k (t))}{\partial t^2} +(k^2c^2_s - 4\pi G\varepsilon_0)\delta\varepsilon_k (t) = -\sigma k^2 \delta S_k
\end{equation} 

A solução geral da equação diferencial pode ser escrita como a soma da solução da equação homogênea com a solução particular independente do tempo da equação (30)
$$\dfrac{\partial^2 (\delta\varepsilon_k (t))}{\partial t^2} +(k^2c^2_s - 4\pi G\varepsilon_0)\delta\varepsilon_k (t) = -\sigma k^2 \delta S_k$$
Para a solução independente do tempo $\dfrac{\partial^2 (\delta\varepsilon_k (t))}{\partial t^2} = 0$, então

$$(k^2c^2_s - 4\pi G\varepsilon_0)\delta\varepsilon_k = -\sigma k^2 \delta S_k$$
\begin{equation}
	\delta\varepsilon_k  = \dfrac{ -\sigma k^2 \delta S_k}{k^2c^2_s - 4\pi G\varepsilon_0}
\end{equation}

é chamada de pertubação de entropia. Nota-se que $k to \infty$ quando a gravidade não é relevante, 
$$\lim_{k \to\infty}\delta\varepsilon_k = \lim_{k \to\infty}\dfrac{ -\sigma k^2 \delta S_k}{k^2c^2_s - 4\pi G\varepsilon_0}$$

$$\lim_{k \to\infty}\delta\varepsilon_k = \lim_{k \to\infty}\dfrac{ -\sigma \delta S_k}{c^2_s}$$

$$\lim_{k \to\infty}\delta\varepsilon_k =-\dfrac{ \sigma \delta S_k}{c^2_s}$$
Neste caso, a contribuição para a pressão devido à falta de homogeneidade da densidade de energia é exatamente compensada pela contribuição correspondente das perturbações de entropia, de modo que $\delta p = c_s^2 \delta\varepsilon_k + \sigma\delta S_k$ desaparece. 

As perturbações de entropia podem ocorrer apenas em fluidos multicomponentes. Por exemplo, em um fluido que consiste em bárions e radiação, os bárions podem ser distribuídos de forma não homogênea em um fundo homogêneo de radiação. Nesse caso, a entropia, que é igual ao número de fótons por barião, varia de um lugar para outro.
Assim, encontramos o conjunto completo de modos - dois modos adiabáticos, dois
modos de vetor e um modo de entropia - descrevendo perturbações em uma gravitação
meio não expansível homogêneo. O mais interessante é o exponencialmente
crescente modo adiabático que é responsável pela origem da estrutura no
universo.

\begin{center}
	\textbf{6.3 Instabilidade de um universo em expansão}
\end{center}

Em um universo homogêneo e isotrópico, a densidade de energia é uma função do tempo e a velocidades obedecem a lei de Hubble
\begin{equation}
	\varepsilon = \varepsilon_0 (t), \,\,\, \mathbf{V} = \mathbf{V}_0 = \mathbf{H} (t) \mathbf{x}
\end{equation}

Substituindo essas equações na equação (3), temos
$$\dfrac{\partial \varepsilon_0}{\partial t} +\nabla \,(\varepsilon_0 \, \textbf{V}) = 0$$

$$\dfrac{\partial \varepsilon_0}{\partial t} +\varepsilon_0\nabla \,( \mathbf{H} \mathbf{x}) = 0$$

$$\dfrac{\partial \varepsilon_0}{\partial t} +\varepsilon_0\mathbf{H}\nabla(\mathbf{x}) = 0$$

Para um universo homogêneo e isotrópico temos $\dfrac{\partial\mathbf{x}}{\partial x}=\dfrac{\partial\mathbf{x}}{\partial y} =\dfrac{\partial\mathbf{x}}{\partial z} = 1$

$$\dfrac{\partial \varepsilon_0}{\partial t} +\mathbf{H}\left(\dfrac{\partial\mathbf{x}}{\partial x}+\dfrac{\partial\mathbf{x}}{\partial y} +\dfrac{\partial\mathbf{x}}{\partial z} \right)\varepsilon_0 = 0$$

$$\dfrac{\partial \varepsilon_0}{\partial t} +3\mathbf{H} \dfrac{\partial\mathbf{x}}{\partial x}\varepsilon_0= 0$$

\begin{equation}
	\dot{\varepsilon_0} + 3 \mathbf{H}\varepsilon_0 = 0 
\end{equation}

que afirma matematicamente que a massa com a massa não relativística se conserva. A divergência da equação de Euler (8) juntamente com a equação de Poisson (10) leva a equação de Friedmann 

 $$\dfrac{\partial \textbf{V} }{\partial t} + (\textbf{V} \cdot \nabla ) \textbf{V} + \frac{\nabla p}{\varepsilon} + \nabla \phi =0$$ com $\nabla^2\phi = 4\pi G\varepsilon_0$.

Dado que $\nabla\cdot\mathbf{V} = \nabla \cdot (\mathbf{H}(t)\mathbf{x}) = \mathbf{H}(t) (\nabla\cdot\mathbf{x}) = 3\mathbf{H}(t)$, $\frac{\nabla p}{\varepsilon} = 0$ e

 $(\mathbf{V}\cdot\nabla)\mathbf{V} = (\mathbf{H}\mathbf{x}\cdot\nabla)\mathbf{V}=(\mathbf{H}\mathbf{x}\cdot\dfrac{\partial }{\partial \mathbf{x}})\mathbf{V}= \mathbf{H}\mathbf{V}$, logo


$$\dfrac{\partial \textbf{V} }{\partial t} + ( \nabla\cdot\textbf{V} ) \textbf{V}   + \nabla \phi =0$$

$$\dfrac{\partial \textbf{V} }{\partial t} +\mathbf{H}(t)\mathbf{V}  + \nabla \phi =0$$

 $$\nabla\cdot \left(\dfrac{\partial \textbf{V} }{\partial t} +\mathbf{H}(t)\mathbf{V}  + \nabla \phi \right) =\nabla\cdot0$$
 
$$\nabla\cdot \left(\dfrac{\partial \textbf{V} }{\partial t} \right)+\nabla\cdot \left( \mathbf{H}(t)\mathbf{V} \right)+\nabla\cdot \left(\nabla \phi\right) =0$$

$$ \dfrac{\partial}{\partial t} (\nabla\cdot\textbf{V})  +\nabla\cdot \left( \mathbf{H}(t)\mathbf{V}  \right)+\nabla^2 \phi =0$$	

$$ \dfrac{\partial}{\partial t} (\nabla\cdot\textbf{V})  +\mathbf{H}(t)\nabla\cdot \mathbf{V} +\Delta\phi =0$$

$$ 3\dfrac{\partial\mathbf{H}(t)}{\partial t}  +3\mathbf{H}^2(t) +\Delta\phi =0$$

$$ 3(\dot{\mathbf{H}}  +\mathbf{H}^2(t))  =-\Delta\phi$$

$$ 3(\dot{\mathbf{H}}  +\mathbf{H}^2(t))  =-4\pi G\varepsilon_0$$

\begin{equation}
	\dot{\mathbf{H}} + \mathbf{H}^2 = - \dfrac{4\pi G}{3}\varepsilon_0 
\end{equation}

Pertubações ignorando as pertubações de entropia e substituindo as expressões, obtemos
\begin{equation}
	\varepsilon = \varepsilon_0 + \delta\varepsilon_0 (\mathbf{x},t),\,\, \mathbf{V} = \mathbf{V}_0 + \delta\mathbf{v} , \phi= \phi_0 + \delta\phi\,\,\, 
\end{equation}

$$p = p_0 + \delta p= p_0 + c_2^2\delta\varepsilon$$

aplicando nas equações (3), (8) e (10), encontramos as equações lineares para pequenas pertubações
$$\dfrac{\partial \varepsilon}{\partial t} +\nabla \,(\varepsilon \, \textbf{V}) = 0$$

$$\dfrac{\partial (\varepsilon_0 + \delta\varepsilon_0)}{\partial t} +\nabla \,( (\varepsilon_0 + \delta\varepsilon_0) \, (\mathbf{V}_0 + \delta\mathbf{v})) = 0$$

$$\dfrac{\partial \varepsilon_0}{\partial t} +\dfrac{\partial \delta\varepsilon_0}{\partial t} +\nabla \,( (\varepsilon_0 + \delta\varepsilon_0) \, \mathbf{V}_0) +\nabla \,( (\varepsilon_0 + \delta\varepsilon_0) \, \delta\mathbf{v})= 0$$

$$\dfrac{\partial \varepsilon_0}{\partial t} +\dfrac{\partial \delta\varepsilon_0}{\partial t} +\nabla \,( (\varepsilon_0 + \delta\varepsilon_0) \, \mathbf{V}_0) +\nabla \,( (\varepsilon_0 + \delta\varepsilon_0) \, \delta\mathbf{v})= 0$$

$$\dfrac{\partial \varepsilon_0}{\partial t} +\dfrac{\partial \delta\varepsilon_0}{\partial t} +\varepsilon_0 \nabla\mathbf{V}_0+\nabla \,( \delta\varepsilon_0  \cdot \mathbf{V}_0) +\varepsilon_0\nabla \delta\mathbf{v}+\nabla \, ( \delta\varepsilon_0 \cdot \delta\mathbf{v})= 0$$

Desconsiderando os termos não relacionados há pequenas pertubações temos
$$\dfrac{\partial \delta\varepsilon_0}{\partial t} +\nabla \,( \delta\varepsilon_0  \cdot \mathbf{V}_0) +\varepsilon_0\nabla \delta\mathbf{v}+\nabla \, ( \delta\varepsilon_0 \cdot \delta\mathbf{v})= 0$$

com  $\nabla \, ( \delta\varepsilon_0 \cdot \delta\mathbf{v}) \to 0$, temos

\begin{equation}
	\dfrac{\partial\delta\varepsilon}{\partial t} + \varepsilon_0(\nabla\cdot\delta\mathbf{v})+\mathbf{\nabla} (\delta\varepsilon \cdot \mathbf{V}) = 0
\end{equation}
\newline

$$\dfrac{\partial \textbf{V} }{\partial t} + (\textbf{V} \cdot \nabla ) \textbf{V} + \frac{\nabla p}{\varepsilon} + \nabla \phi =0$$

$$\dfrac{\partial (\mathbf{V}_0 + \delta\mathbf{v}) }{\partial t} + ((\mathbf{V}_0 + \delta\mathbf{v}) \cdot \nabla ) (\mathbf{V}_0 + \delta\mathbf{v}) + \frac{\nabla (p_0 + c_s^2\delta\varepsilon)}{\varepsilon_0} + \nabla (\phi_0 + \delta\phi) =0$$

$$\dfrac{\partial \mathbf{V}_0 }{\partial t} +\dfrac{\partial \delta\mathbf{v} }{\partial t}+ ((\mathbf{V}_0 + \delta\mathbf{v}) \cdot \nabla )\mathbf{V}_0+ ((\mathbf{V}_0 + \delta\mathbf{v}) \cdot \nabla )\delta\mathbf{v} + \frac{\nabla p_0}{\varepsilon_0}+ \frac{c_s^2\nabla\delta\varepsilon}{\varepsilon_0} + \nabla \phi_0+ \nabla\delta\phi =0$$

$$\dfrac{\partial \mathbf{V}_0 }{\partial t} +\dfrac{\partial \delta\mathbf{v} }{\partial t}+ (\mathbf{V}_0\cdot \nabla + \delta\mathbf{v}\cdot \nabla)\mathbf{V}_0+ (\mathbf{V}_0\cdot \nabla + \delta\mathbf{v}\cdot \nabla) \delta\mathbf{v} + \frac{\nabla p_0}{\varepsilon_0}+ \frac{c_s^2\nabla\delta\varepsilon}{\varepsilon_0} + \nabla \phi_0+ \nabla\delta\phi =0$$

$\displaystyle\dfrac{\partial \mathbf{V}_0 }{\partial t} +\dfrac{\partial \delta\mathbf{v} }{\partial t}+ (\mathbf{V}_0\cdot \nabla)\mathbf{V}_0+ (\delta\mathbf{v}\cdot \nabla)\mathbf{V}_0+ (\mathbf{V}_0\cdot \nabla) \delta\mathbf{v} + (\delta\mathbf{v}\cdot \nabla) \delta\mathbf{v} + \frac{\nabla p_0}{\varepsilon_0}+ \frac{c_s^2\nabla\delta\varepsilon}{\varepsilon_0} + \nabla \phi_0+ \nabla\delta\phi =0$

Desconsiderando os termos não relacionados há pequenas pertubações temos

$$\dfrac{\partial \delta\mathbf{v} }{\partial t}+ (\delta\mathbf{v}\cdot \nabla)\mathbf{V}_0+ (\mathbf{V}_0\cdot \nabla) \delta\mathbf{v} + (\delta\mathbf{v}\cdot \nabla) \delta\mathbf{v} + \frac{c_s^2\nabla\delta\varepsilon}{\varepsilon_0}+ \nabla\delta\phi =0$$

com $(\delta\mathbf{v}\cdot \nabla) \delta\mathbf{v} \to 0$, obtemos
\begin{equation}
	\dfrac{\partial\delta\mathbf{v}}{\partial t} + (\mathbf{V}_0 \cdot \nabla)\delta\mathbf{v} + (\delta\mathbf{v} \cdot \nabla) \mathbf{V}_0 + \dfrac{c^2_s}{\varepsilon_0}\nabla\delta\varepsilon + \nabla\delta\phi = 0
\end{equation}
\newline
e
\begin{equation}
	\Delta\delta\phi = 4\pi G\delta\varepsilon.
\end{equation}

A velocidade de Hubble $\mathbf{V}_0$ depende explicitamente de $\mathbf{x}$ e, portanto, da transformada de Fourier em relação às coordenadas eulerianas $\mathbf{x}$ não reduz essas equações a um conjunto desacoplado de equações diferenciais ordinárias. É por isso que é mais conveniente para usar as coordenadas Lagrangianas (comovendo com o fluxo de Hubble) $\mathbf{q}$, que são relacionado com as coordenadas Eulerianas via

\begin{equation}
	\mathbf{x} = a(t)\mathbf{q}
\end{equation}

onde $a(t)$ é um fator de escala. A derivada parcial em relação ao tempo tomado em
constante $\mathbf{x}$ é diferente da derivada parcial tomada na constante $\mathbf{q}$. Para um general função $f (\mathbf{x}, t)$ que temos

\begin{equation}
	\left( \dfrac{\partial f(\mathbf{x} = a(t)\mathbf{q}\, , t)}{\partial t} \right)_\mathbf{q} = \left( \dfrac{\partial f}{\partial t} \right)_\mathbf{x} + \dot{a}q^i\left( \dfrac{\partial f}{\partial x^i} \right)_t
\end{equation}
$$\left( \dfrac{\partial f(\mathbf{x} = a(t)\mathbf{q}\, , t)}{\partial t} \right)_\mathbf{q} = \left( \dfrac{\partial f}{\partial t} \right)_\mathbf{x} + \dfrac{\partial\mathbf{x}}{\partial t}\left( \dfrac{\partial f}{\partial x^i} \right)_t$$

$$\left( \dfrac{\partial f(\mathbf{x} = a(t)\mathbf{q}\, , t)}{\partial t} \right)_\mathbf{q} = \left( \dfrac{\partial f}{\partial t} \right)_\mathbf{x} + \mathbf{V}_0\cdot\nabla_\mathbf{x} f$$

$$\left(\left( \dfrac{\partial }{\partial t} \right)_\mathbf{q}\right)f =\left(\left( \dfrac{\partial }{\partial t} \right)_\mathbf{x} + \mathbf{V}_0\cdot\nabla_\mathbf{x}\right)  f$$

$$\left( \dfrac{\partial }{\partial t} \right)_\mathbf{q} =\left( \dfrac{\partial }{\partial t} \right)_\mathbf{x} + \mathbf{V}_0\cdot\nabla_\mathbf{x}$$
e portanto

\begin{equation}
	\left(\dfrac{\partial}{\partial t}\right)_\mathbf{x} = \left( \dfrac{\partial}{\partial t} \right)_\mathbf{q} - (\mathbf{V}_0 \cdot \nabla_\mathbf{x})
\end{equation}

As derivadas espaciais estão relacionadas de forma mais simples

\begin{equation}
	\nabla_\mathbf{x} = \dfrac{1}{a}\nabla_\mathbf{q}
\end{equation}

Substituindo as derivadas em (36)-(38) e introduzindo a amplitude fracionária das pertubações de densidade $\delta\equiv\dfrac{\delta\varepsilon}{\varepsilon_0}$, e finalmente obtemos

$$\dfrac{\partial\delta\varepsilon}{\partial t} + \varepsilon_0(\nabla\cdot\delta\mathbf{v})+\mathbf{\nabla} (\delta\varepsilon \cdot \mathbf{V}) = 0$$

$$\left(\dfrac{\partial(\varepsilon_0\delta)}{\partial t}\right) + \varepsilon_0\nabla_\mathbf{x}\delta\mathbf{v}+\mathbf{\nabla} ((\varepsilon_0\delta) \cdot \mathbf{V}) = 0$$

$$\varepsilon_0\left(\dfrac{\partial(\delta)}{\partial t}\right) + \varepsilon_0\nabla_\mathbf{x}\delta\mathbf{v}+\varepsilon_0\mathbf{\nabla} ((\delta) \cdot \mathbf{V}) = 0$$

$$\left(\dfrac{\partial\delta}{\partial t}\right) + \nabla_\mathbf{x}\delta\mathbf{v} = 0$$

$$\left(\dfrac{\partial\delta}{\partial t}\right) + \dfrac{1}{a}\nabla_\mathbf{q}\delta\mathbf{v} = 0$$

\begin{equation}
	\left( \dfrac{\partial \delta}{\partial t} \right) + \dfrac{1}{a}(\nabla\cdot\delta\mathbf{v}) = 0
\end{equation}
\newline


$$\dfrac{\partial\delta\mathbf{v}}{\partial t} + (\mathbf{V}_0 \cdot \nabla)\delta\mathbf{v} + (\delta\mathbf{v} \cdot \nabla) \mathbf{V}_0 + \dfrac{c^2_s}{\varepsilon_0}\nabla\delta\varepsilon + \nabla\delta\phi = 0$$

$$\left( \dfrac{\partial\delta\mathbf{v}}{\partial t} \right) + (\mathbf{V}_0 \cdot \nabla)\delta\mathbf{v} + (\delta\mathbf{v} \cdot \nabla) \mathbf{V}_0 + \dfrac{c^2_s}{\varepsilon_0}\nabla\delta\varepsilon + \nabla\delta\phi = 0$$

$$\left( \dfrac{\partial\delta\mathbf{v}}{\partial t} \right) +H\delta\mathbf{v} + (\delta\mathbf{v} \cdot \nabla_\mathbf{x}) \mathbf{V}_0 + \dfrac{c^2_s}{\varepsilon_0}\nabla_\mathbf{x}\delta\varepsilon + \nabla_\mathbf{x}\delta\phi = 0$$

$$\left( \dfrac{\partial\delta\mathbf{v}}{\partial t} \right) +H\delta\mathbf{v} + c^2_s\nabla_\mathbf{x}\dfrac{\delta\varepsilon}{\varepsilon_0} + \nabla_\mathbf{x}\delta\phi = 0$$

$$\left( \dfrac{\partial\delta\mathbf{v}}{\partial t} \right) +H\delta\mathbf{v} + \dfrac{c^2_s}{a}\nabla_\mathbf{q}\delta + \dfrac{1}{a}\nabla_\mathbf{q}\delta\phi = 0$$

\begin{equation}
	\left( \dfrac{\partial \delta\mathbf{v}}{\partial t} \right) +H\delta\mathbf{v}+\dfrac{c^2_s}{a}\nabla\delta + \dfrac{1}{a}\nabla\delta\phi =0
\end{equation}
\newline


$$\Delta\delta\phi = 4\pi G\delta\varepsilon$$

$$\nabla^2_\mathbf{x}\delta\phi = 4\pi G(\varepsilon_0\delta)$$

$$\dfrac{\nabla^2_\mathbf{q}\delta\phi}{a^2} = 4\pi G(\varepsilon_0\delta)$$

\begin{equation}
	\Delta\delta\phi = 4\pi Ga^2\varepsilon_0\delta
\end{equation}

onde $\nabla\equiv\nabla_\mathbf{q}$ e $\Delta$ agora são as derivadas em relação às coordenadas de Lagrange $\mathbf{q}$ e as derivadas de tempo são tomadas na constante $\mathbf{q}$. Ao derivar (43), temos usado (33) para o fundo e observou que $\nabla\cdot\mathbf{V}_0 = 3\mathbf{H}(t)$ e $(\delta\mathbf{v}\cdot\nabla\mathbf{x}) \mathbf{V}_0 = H\delta\mathbf{v}$.
Tomando a divergência de (44) e usando as equações de continuidade e Poisson para
expressar $(\nabla\cdot\delta\mathbf{v})$ e $\delta\phi$ em termos de $\delta$, derivamos a equação de forma fechada.

Com as equações (43) podemos obter as seguintes relações
$$\left( \dfrac{\partial \delta}{\partial t} \right) + \dfrac{1}{a}\nabla_\mathbf{q}\delta\mathbf{v} = 0$$

$$\left( \dfrac{\partial \delta}{\partial t} \right)   = -\dfrac{1}{a}\nabla_\mathbf{q}\delta\mathbf{v}$$
\newline

$$\dfrac{\partial}{\partial t}\left( \dfrac{\partial \delta}{\partial t}  + \dfrac{1}{a}\nabla_\mathbf{q}\delta\mathbf{v} \right)= 0$$

$$\dfrac{\partial^2 \delta}{\partial t^2} + \dfrac{\partial}{\partial t}\left( \dfrac{1}{a}\nabla_\mathbf{q}\delta\mathbf{v} \right)= 0$$

$$\dfrac{\partial^2 \delta}{\partial t^2} + \dfrac{ \dfrac{1}{a}\dfrac{\partial}{\partial t }(\nabla_\mathbf{q}\cdot\delta\mathbf{v})- \dfrac{\partial a}{\partial t}(\nabla_\mathbf{q}\cdot\delta\mathbf{v})}{a^2} = 0$$

como $\dfrac{\partial a}{\partial t}=a\mathbf{H}$, portanto

$$\dfrac{\partial^2 \delta}{\partial t^2} +  \dfrac{1}{a}\dfrac{\partial}{\partial t }(\nabla_\mathbf{q}\cdot\delta\mathbf{v})-  \dfrac{a\mathbf{H}}{a^2}(\nabla_\mathbf{q}\cdot\delta\mathbf{v}) = 0$$

$$\dfrac{1}{a}\dfrac{\partial}{\partial t }(\nabla_\mathbf{q}\cdot\delta\mathbf{v})= -\dfrac{\partial^2 \delta}{\partial t^2} + \dfrac{\mathbf{H}}{a}(\nabla_\mathbf{q}\cdot\delta\mathbf{v})$$

$$\dfrac{1}{a}\dfrac{\partial}{\partial t }(\nabla_\mathbf{q}\cdot\delta\mathbf{v})= -\dfrac{\partial^2 \delta}{\partial t^2} -\mathbf{H} \dfrac{\partial\delta}{\partial t}$$

com as relações a cima, prosseguimos dessa forma
$$\nabla_\mathbf{x}\cdot\left( \dfrac{\partial \delta\mathbf{v}}{\partial t}  +\mathbf{H}\delta\mathbf{v}+\dfrac{c^2_s}{a}\nabla_\mathbf{x}\delta + \dfrac{1}{a}\nabla_\mathbf{q}\delta\phi \right)=\nabla_\mathbf{x}\cdot0$$

$$\nabla_\mathbf{x}\cdot\left( \dfrac{\partial \delta\mathbf{v}}{\partial t}  \right)+\nabla_\mathbf{x}\cdot\left( \mathbf{H}\delta\mathbf{v}\right)+\nabla_\mathbf{x}\cdot\left( \dfrac{c^2_s}{a}\nabla_\mathbf{q}\delta\right)+\nabla_\mathbf{x}\cdot\left(  \dfrac{1}{a}\nabla_\mathbf{x}\delta\phi \right)=0$$

$$\dfrac{1}{a}\dfrac{\partial (\nabla_\mathbf{q}\cdot\delta\mathbf{v})}{\partial t}  + \dfrac{\mathbf{H}}{a}(\nabla_\mathbf{q}\cdot\delta\mathbf{v}) + \dfrac{c^2_s}{a^2}\nabla^2_\mathbf{q}\delta+ \dfrac{1}{a^2}\nabla^2_\mathbf{q}\delta\phi =0$$

$$ -\dfrac{\partial^2 \delta}{\partial t^2} -\mathbf{H} \dfrac{\partial\delta}{\partial t}  -\mathbf{H}\dfrac{\partial\delta}{\partial t} + \dfrac{c^2_s}{a^2}\nabla^2_\mathbf{q}\delta+ \dfrac{1}{a^2}(4\pi Ga^2\varepsilon_0\delta)=0$$

$$ -\dfrac{\partial^2 \delta}{\partial t^2} -2\mathbf{H} \dfrac{\partial\delta}{\partial t} + \dfrac{c^2_s}{a^2}\nabla^2_\mathbf{q}\delta+ 4\pi G\varepsilon_0\delta=0$$

\begin{equation}
	\ddot{\delta} + 2\mathbf{H}\dot{\delta} - \dfrac{c^2_s}{a^2}\Delta\delta - 4\pi G\varepsilon_0\delta = 0
\end{equation}
que descreve a instabilidade gravitacional em um universo em expansão.


\begin{center}
	\textbf{6.3.1 Pertubações Adiabáticas}
\end{center}
Tomando a transformada de Fourier da equação (46) para as coordenadas comoventes $\mathbf{q}$,

$$\delta = \int \delta_\mathbf{k} (t)e^{i\mathbf{k}\mathbf{q}}\dfrac{d^3k}{(2\pi)^{3/2}}$$

$\displaystyle\dfrac{\partial^2}{\partial t^2}\left(\int \delta_\mathbf{k} (t)e^{i\mathbf{k}\mathbf{q}}\dfrac{d^3k}{(2\pi)^{3/2}}\right) + 2\mathbf{H}\dfrac{\partial}{\partial t}\left(\int \delta_\mathbf{k} (t)e^{i\mathbf{k}\mathbf{q}}\dfrac{d^3k}{(2\pi)^{3/2}}\right) - \dfrac{c^2_s}{a^2}\Delta\left(\int \delta_\mathbf{k} (t)e^{i\mathbf{k}\mathbf{q}}\dfrac{d^3k}{(2\pi)^{3/2}}\right)- 4\pi G\varepsilon_0\left(\int \delta_\mathbf{k} (t)e^{i\mathbf{k}\mathbf{q}}\dfrac{d^3k}{(2\pi)^{3/2}}\right) $

$=0$

$$\int \left(\dfrac{\partial^2\delta_\mathbf{k}(t)}{\partial t^2}+ 2\mathbf{H}\dfrac{\partial\delta_\mathbf{k}(t)}{\partial t}  - 4\pi G\varepsilon_0\delta_\mathbf{k}(t) \right) e^{i\mathbf{k}\mathbf{q}}\dfrac{d^3k}{(2\pi)^{3/2}} - \dfrac{c^2_s}{a^2}\left(\int \Delta\delta_\mathbf{k} (t)e^{i\mathbf{k}\mathbf{q}}\dfrac{d^3k}{(2\pi)^{3/2}}\right)=0 $$

$$\int \left(\dfrac{\partial^2\delta_\mathbf{k}(t)}{\partial t^2}+ 2\mathbf{H}\dfrac{\partial\delta_\mathbf{k}(t)}{\partial t}  - 4\pi G\varepsilon_0\delta_\mathbf{k}(t) \right) e^{i\mathbf{k}\mathbf{q}}\dfrac{d^3k}{(2\pi)^{3/2}} - \dfrac{c^2_s(|k|i)^2}{a^2}\left(\int \delta_\mathbf{k} (t)e^{i\mathbf{k}\mathbf{q}}\dfrac{d^3k}{(2\pi)^{3/2}}\right)=0 $$

$$\int \left(\dfrac{\partial^2\delta_\mathbf{k}(t)}{\partial t^2}+ 2\mathbf{H}\dfrac{\partial\delta_\mathbf{k}(t)}{\partial t}  - 4\pi G\varepsilon_0\delta_\mathbf{k}(t) \right) e^{i\mathbf{k}\mathbf{q}}\dfrac{d^3k}{(2\pi)^{3/2}} + \dfrac{c^2_sk^2}{a^2}\left(\int \delta_\mathbf{k} (t)e^{i\mathbf{k}\mathbf{q}}\dfrac{d^3k}{(2\pi)^{3/2}}\right)=0 $$

$$\int \left(\dfrac{\partial^2\delta_\mathbf{k}(t)}{\partial t^2}+ 2\mathbf{H}\dfrac{\partial\delta_\mathbf{k}(t)}{\partial t} + \dfrac{c^2_sk^2}{a^2}\delta_\mathbf{k}(t) - 4\pi G\varepsilon_0\delta_\mathbf{k}(t) \right) e^{i\mathbf{k}\mathbf{q}}\dfrac{d^3k}{(2\pi)^{3/2}} =0 $$

$$\int \left(\dfrac{\partial^2\delta_\mathbf{k}(t)}{\partial t^2}+ 2\mathbf{H}\dfrac{\partial\delta_\mathbf{k}(t)}{\partial t} +( \dfrac{c^2_sk^2}{a^2} - 4\pi G\varepsilon_0)\delta_\mathbf{k}(t) \right) e^{i\mathbf{k}\mathbf{q}}\dfrac{d^3k}{(2\pi)^{3/2}} =0 $$

multiplicando por $e^{i\mathbf{k}^\prime\mathbf{q}}$
$$e^{i\mathbf{k}^\prime\mathbf{q}}\left(\int \left(\dfrac{\partial^2\delta_\mathbf{k}(t)}{\partial t^2}+ 2\mathbf{H}\dfrac{\partial\delta_\mathbf{k}(t)}{\partial t} +( \dfrac{c^2_sk^2}{a^2} - 4\pi G\varepsilon_0)\delta_\mathbf{k}(t) \right) e^{i\mathbf{k}\mathbf{q}}\dfrac{d^3k}{(2\pi)^{3/2}}\right) =0e^{i\mathbf{k}^\prime\mathbf{q}} $$

$$\int \left(\dfrac{\partial^2\delta_\mathbf{k}(t)}{\partial t^2}+ 2\mathbf{H}\dfrac{\partial\delta_\mathbf{k}(t)}{\partial t} +( \dfrac{c^2_sk^2}{a^2} - 4\pi G\varepsilon_0)\delta_\mathbf{k}(t) \right) e^{i\mathbf{q}(\mathbf{k}+\mathbf{k}^\prime)}\dfrac{d^3k}{(2\pi)^{3/2}} =0 $$
realizando a integral definida de $\mathbf{q}$, temos

$$\int\left( \int \left(\dfrac{\partial^2\delta_\mathbf{k}(t)}{\partial t^2}+ 2\mathbf{H}\dfrac{\partial\delta_\mathbf{k}(t)}{\partial t} +( \dfrac{c^2_sk^2}{a^2} - 4\pi G\varepsilon_0)\delta_\mathbf{k}(t) \right) e^{i\mathbf{q}(\mathbf{k}+\mathbf{k}^\prime)}\dfrac{d^3k}{(2\pi)^{3/2}} \right)d\mathbf{q} =\int0d\mathbf{q} $$

$$ \int \left(\dfrac{\partial^2\delta_\mathbf{k}(t)}{\partial t^2}+ 2\mathbf{H}\dfrac{\partial\delta_\mathbf{k}(t)}{\partial t} +( \dfrac{c^2_sk^2}{a^2} - 4\pi G\varepsilon_0)\delta_\mathbf{k}(t) \right) \int\left(e^{i\mathbf{q}(\mathbf{k}+\mathbf{k}^\prime)}\dfrac{d\mathbf{q}}{(2\pi)^{3/2}} \right)d^3k =0 $$

$$ \int \left(\dfrac{\partial^2\delta_\mathbf{k}(t)}{\partial t^2}+ 2\mathbf{H}\dfrac{\partial\delta_\mathbf{k}(t)}{\partial t} +( \dfrac{c^2_sk^2}{a^2} - 4\pi G\varepsilon_0)\delta_\mathbf{k}(t) \right) \delta(\mathbf{k}+\mathbf{k}^\prime)d^3k =0 $$

$$\int\delta(\mathbf{k}+\mathbf{k}^\prime)\left(\dfrac{\partial^2\delta_\mathbf{k}(t)}{\partial t^2}+ 2\mathbf{H}\dfrac{\partial\delta_\mathbf{k}(t)}{\partial t} +( \dfrac{c^2_sk^2}{a^2} - 4\pi G\varepsilon_0)\delta_\mathbf{k}(t) \right) d^3k =0 $$
sabendo que o integrando deve ser equivalente a zero para que o resultado da integração ser igual a zero, assim obtemos a equação diferencial ordinária
 
\begin{equation}
	\ddot{\delta_\mathbf{k}} + 2\mathbf{H}\dot{\delta_\mathbf{k}} + \left( \dfrac{c^2_s k^2}{a^2} -4\pi G\varepsilon_0\right)\delta_\mathbf{k} = 0
\end{equation}

O comportamento de cada perturbação depende crucialmente de seu tamanho espacial; a escala de comprimento crítica é o comprimento de jeans

\begin{equation}
	\lambda^{ph}_J = \dfrac{2\pi a }{k_J} = c_s\sqrt{\dfrac{\pi}{G\varepsilon_0}}
\end{equation}

Aqui $\lambda^{ph}$ é o comprimento de onda físico, relacionado
ao comprimento de onda comovente $\lambda = \dfrac{2\pi}{k}$ via $\lambda^{ph} = a\lambda$. Em um universo plano dominado pela matéria $\varepsilon_0 = (6\pi G t^2)^{-1}$ e, portanto,

\begin{equation}
	\lambda_J^{ph} \sim c_s t,
\end{equation}

isto é, o comprimento do jeans está na ordem do horizonte de som. Às vezes, em vez do
Comprimento de jeans, usa-se a massa de jeans, definida como $M_J \equiv\varepsilon_0 (\lambda_J^{ph})^3$.
Perturbações em escalas muito menores do que o comprimento de jeans ($\lambda \ll\lambda_J$) são válidas ondas. Se $c_s$ mudar adiabaticamente, então a solução de (47) é

\begin{equation}
	\delta_\mathbf{k} \propto \dfrac{1}{\sqrt{c_s a}} \,exp\left( \pm k \int \dfrac{c_s dt}{a}\right)
\end{equation}

Em escalas muito maiores do que a escala de Jeans ($\lambda \gg \lambda_J$), a gravidade domina e nós
pode negligenciar o termo dependente de k em (47). Então, uma das soluções é simplesmente
proporcional à constante de Hubble $H (t)$. Na verdade, substituindo $\delta_d = H (t)$ em (47),
onde definimos $c^2_s k^2 = 0$, descobre-se que a equação resultante coincide com o tempo
derivada da equação de Friedmann (34). Observe que $\delta_d = H (t)$ é o decaimento
solução da equação de perturbação (H diminui com o tempo) em um dominado pela matéria
universo com curvatura arbitrária.

Tanto a densidade de energia de fundo $\varepsilon_0 (t)$ quanto a energia deslocada no tempo
densidade $\varepsilon_0 (t + \tau)$, onde $\tau =$ const, satisfaça (33), (34). Na verdade, usando (33)
para expressar H em termos de $\varepsilon_0$ e substituí-lo em (34), obtemos uma equação para $\varepsilon_0 (t)$ em que o tempo não aparece explicitamente. Portanto, sua solução é invariante no tempo e invariante a translação. Para $\tau$ pequeno, a solução deslocada no tempo $\varepsilon_0 (t+ \tau)$ pode ser
considerado como uma perturbação do fundo $\varepsilon_0 (t)$ com amplitude

$$\delta_d = \dfrac{\varepsilon_0 (t+ \tau) -\varepsilon_0 (t) }{\varepsilon_0 (t)} \approx \dfrac{\dot{\varepsilon_0} (t)\tau }{\varepsilon_0 (t)} \propto H(t)$$.

Uma vez que conhecemos uma solução da equação diferencial de segunda ordem, $\delta_d$, então o
outra solução independente $\delta_i$ pode ser facilmente encontrada com a ajuda do Wronskiano

\begin{equation}
	W \equiv \dot{\delta_d} \delta_i - \dot{\delta_i} \delta_d
\end{equation}

Tomando a derivada do Wronskiano e usando (47) para expressar $\ddot{\delta}$ em termos de $\dot{\delta}$ e $\delta$, descobrimos que $W$ satisfaz a equação


$$W \equiv \dot{\delta_d} \delta_i - \dot{\delta_i} \delta_d$$

$$\dfrac{\partial W }{\partial t}= \dfrac{\partial  }{\partial t}\left(\dot{\delta_d} \delta_i - \dot{\delta_i} \delta_d \right)$$

$$\dfrac{\partial W }{\partial t}= \dfrac{\partial  }{\partial t}\left(\dot{\delta_d} \delta_i  \right)-\dfrac{\partial  }{\partial t}\left(\dot{\delta_i} \delta_d \right)$$

$$\dot{W}= \delta_i\dfrac{\partial  \dot{\delta_d}}{\partial t}+\dot{\delta_d}\dfrac{\partial \delta_i }{\partial t}-\delta_d\dfrac{\partial  \dot{\delta_i} }{\partial t}-\dot{\delta_i}\dfrac{\partial \delta_d }{\partial t}$$

$$\dot{W}= \delta_i\ddot{\delta_d}-\delta_d\ddot{\delta_i}$$

Sabendo que $\delta_d$ e $\delta_i$ satisfazem a equação (46), isto é,
$$\ddot{\delta_d} + 2\mathbf{H}\dot{\delta_d} + \left( \dfrac{c^2_s k^2}{a^2} -4\pi G\varepsilon_0\right)\delta_d = 0 $$
e
$$\ddot{\delta_i} + 2\mathbf{H}\dot{\delta_i} + \left( \dfrac{c^2_s k^2}{a^2} -4\pi G\varepsilon_0\right)\delta_i = 0$$
portanto substituindo 
$$\ddot{\delta_d} = - 2\mathbf{H}\dot{\delta_d} - \left( \dfrac{c^2_s k^2}{a^2} -4\pi G\varepsilon_0\right)\delta_d  $$

$$\ddot{\delta_i}  = - 2\mathbf{H}\dot{\delta_i} - \left( \dfrac{c^2_s k^2}{a^2} -4\pi G\varepsilon_0\right)\delta_i$$
temos
$$\dot{W}= \delta_i\left(- 2\mathbf{H}\dot{\delta_d} - \left( \dfrac{c^2_s k^2}{a^2} -4\pi G\varepsilon_0\right)\delta_d  \right)-\delta_d\left( - 2\mathbf{H}\dot{\delta_i} - \left( \dfrac{c^2_s k^2}{a^2} -4\pi G\varepsilon_0\right)\delta_i\right)$$

$$\dot{W}=  -2\mathbf{H}\delta_i\dot{\delta_d} +2\mathbf{H}\delta_d \dot{\delta_i} $$

\begin{equation}
	\dot{W} = -2H W
\end{equation}

que possui a seguinte solução

\begin{equation}
	W \equiv \dot{\delta_d} \delta_i - \dot{\delta_i} \delta_d = \dfrac{C}{a^2}
\end{equation}

onde C é uma constante de integração. Substituindo o $\delta_i = \delta_d f (t)$ em (53), obtemos uma equação para $f$ que é prontamente integrada:

$$\dot{\delta_d} (\delta_d f(t)) - \dfrac{\partial(\delta_d f(t))}{\partial t} \delta_d = \dfrac{C}{a^2}$$

$$\dot{\delta_d}f(t) - \delta_d\dfrac{\partial f(t)}{\partial t} -f(t)\dfrac{\partial\delta_d }{\partial t}= \dfrac{C}{a^2\delta_d}$$

$$- \delta_d\dfrac{\partial f(t)}{\partial t} = \dfrac{C}{a^2\delta_d}$$

\begin{equation}
	f = - \int \dfrac{C}{a^2\delta_d^2}dt.
\end{equation}
Assim, a solução de onda longa mais geral de (47) é

\begin{equation}
	\delta = C_1 H \int \dfrac{dt}{a^2H^2} + C_2 H.
\end{equation}

Em um universo plano dominado pela matéria, $a \propto t^{2/3}$e $H \propto t^{-1}$. Neste caso, temos

\begin{equation}
	\delta = C_1 t^{2/3} + C_2 t^{-1}.
\end{equation}

Portanto, vemos que em um universo em expansão, a instabilidade gravitacional é muito menor
eficiente e a amplitude da perturbação aumenta apenas com a potência do tempo. No caso importante de um universo plano, dominado pela matéria, o modo de crescimento é proporcional
ao fator de escala. Portanto, se quisermos obter grandes inomogeneidades ($\delta \gtrsim 1$) hoje, temos que assumir que nos primeiros tempos (por exemplo, em redshifts $z = 1000$)
as inomogeneidades já eram substanciais ($\delta \gtrsim 10^{-3}$). Isso impõe fortes
restrições no espectro inicial de perturbações.
\newline

\begin{center}
	\textbf{Vetor de Pertubações}
\end{center}
Substituindo $\delta =0 $ nas equações (43)-(45), obtemos
$$\dfrac{\partial (0)}{\partial t} + \dfrac{1}{a}\nabla\delta\mathbf{v} = 0$$

$$\nabla\delta\mathbf{v} = 0$$
\newline

$$\Delta\delta\phi = 4\pi Ga^2\varepsilon_0 (0)$$

$$\Delta\delta\phi = 0$$
\newline

$$\dfrac{\partial\delta\mathbf{v}}{\partial t} + H\delta\mathbf{v} + \dfrac{c^2_s}{a}\nabla(0) +\dfrac{1}{a}(0) = 0$$

$$\dfrac{\partial\delta\mathbf{v}}{\partial t} + H\delta\mathbf{v} = 0$$
\newline

\begin{equation}
	\nabla\delta\mathbf{v} = 0 , \quad \dfrac{\partial\delta\mathbf{v}}{\partial t } + H\delta\mathbf{v} = 0
\end{equation}

Da primeira equação segue que para uma perturbação de onda plana, $\delta\mathbf{v} \propto \delta\mathbf{v}_\mathbf{k} (t)e^{(i\mathbf{kq})}$, a velocidade peculiar $\delta\mathbf{v}$ é perpendicular ao número de onda $\mathbf{k}$. A segunda equação torna-se

\begin{equation}
	\delta\dot{\mathbf{v}_\mathbf{k}} + \dfrac{\dot{a}}{a}\delta\\mathbf{v}_\mathbf{k}=0,
\end{equation}

e tem a solução $\delta\mathbf{v_k} \propto \frac{1}{a}$. Assim, as perturbações do vetor decaem conforme o
o universo se expande. Essas perturbações podem ter amplitudes significativas no presente
apenas se suas amplitudes iniciais fossem tão grandes que estragassem completamente a isotropia do universo primordial. Em um universo inflacionário, não há espaço para tão grande
perturbações vetoriais primordiais e não desempenham qualquer papel na formação de
a estrutura em grande escala do universo. Perturbações de vetor, no entanto, podem ser
gerado tardiamente, após a estrutura não linear ter sido formada, e pode explicar
a rotação das galáxias.

\begin{center}
	\textbf{6.3.3 Solução Auto-similar}
\end{center}
Para perturbações de grande escala, podemos negligenciar a pressão e as derivadas espaciais
desistir de (46). Neste caso, a solução para as perturbações pode ser escrita diretamente
no espaço de coordenadas

\begin{equation}
	\delta (\mathbf{q},t ) = A(\mathbf{q}) \delta_i (t)+B(\mathbf{q})\delta_d (t),
\end{equation}
onde $\delta_i$ e $\delta_d$ são modos crescente e decrescente, respectivamente. Sem perder
generalidade, podemos definir $\delta_i (t_0) = \delta_d (t_0) = 1$ em algum momento inicial de tempo $t_0$. Se o
distribuição de densidade neste momento é descrita pela função $\delta(\mathbf{q}, t_0)$ e a matéria
está em repouso em relação ao fluxo de Hubble ($\delta\mathbf{v} \propto \delta (\mathbf{q}, t_0) = 0$), então, expressando $A (q)$
e $B (\mathbf{q})$ em termos de $\delta (\mathbf{q}, t0)$, obtemos

\begin{equation}
	\delta (\mathbf{q},t) = \delta (\mathbf{q}, t_0)\left( \dfrac{\delta_i(t)}{1 - (\dot{\delta_i}/\dot{\delta_d})_{t_0}} + \dfrac{\delta_d(t)}{1 - (\dot{\delta_d}/\dot{\delta_i})_{t_0}}\right)
\end{equation}

Neste caso particular, a perturbação preserva sua forma espacial inicial à medida que se desenvolve. Essa solução é considerada semelhante a si mesma.
Genericamente, a forma de não homogeneidade muda. No entanto, em horários tardios ($t \gg t_0$)
quando o modo de crescimento dominar, podemos omitir o segundo termo em (59) e o
a perturbação linear cresce de maneira auto-semelhante.
\newline

\begin{center}
	\textbf{6.3.4 Matéria fria na presença de radiação ou energia escura}
\end{center}

Há evidências convincentes de que junto com a matéria fria do universo lá
existe um componente de energia escura suave. Esta energia escura muda a taxa de expansão e, como resultado, influência o crescimento de heterogeneidades na matéria fria.
Para estudar a instabilidade gravitacional na presença de matéria relativística, em princípio, precisamos da teoria relativística completa. No entanto, em escalas menores do que o comprimento jeans
para a matéria relativística, que é comparável à escala do horizonte, as inomogeneidades na distribuição da matéria fria não perturbam o componente relativístico.
e permanece praticamente homogêneo. Como resultado, ainda podemos aplicar modificações da
teoria Newtoniana às perturbações na própria matéria fria em escalas menores
do que o horizonte. A seguir, consideramos o crescimento das perturbações na
presença de um componente de energia relativística homogênea. Isso pode ser radiação,
com equação de estado $w = 1/3$, ou matéria escura com $w < - 1/3$.

É fácil verificar que a equação para a perturbação na componente fria
sozinho, $\delta \equiv \delta\varepsilon_d / \varepsilon_d$, coincide com (46), mas a constante de Hubble agora é determinada
pela densidade total de energia

\begin{equation}
	\varepsilon_{tot} = \dfrac{\varepsilon_{eq}}{2}\left( \left( \dfrac{a_{eq}}{a}\right)^3 + \left(\dfrac{a_{eq}}{a} \right)^{3 (1+w)}\right),
\end{equation}

através da relação usual, que para um universo plano é 
\begin{equation}
	H^2 = \dfrac{8\pi G}{3}\varepsilon_{tot}.
\end{equation}
Aqui um $a_{eq}$ é o fator de escala em “igualdade” quando as densidades de energia de ambos as componentes são iguais. Para encontrar as soluções explícitas de (46), é conveniente reescrevê-lo
usando como uma variável de tempo o fator de escala normalizado $x \equiv a / a_{eq}$ em vez de $t$. Tirando
em consideração que $\varepsilon_0$ entrando em (46) é a densidade de matéria fria sozinha, igual a

\begin{equation}
	\varepsilon_d = \dfrac{\varepsilon_{eq}}{2}\left(\dfrac{a_{eq}}{a}\right)^3.
\end{equation}
e usando (62) para expressar o parâmetro de Hubble em termos de $x$, temos

$$H^2 = \dfrac{8\pi G}{3}\varepsilon_{tot}$$

$$H^2 = \dfrac{8\pi G}{3}\left(\dfrac{\varepsilon_{eq}}{2}\left( \left( \dfrac{a_{eq}}{a}\right)^3 + \left(\dfrac{a_{eq}}{a} \right)^{3 (1+w)}\right)\right)$$

$$H^2 = \dfrac{8\pi G\varepsilon_{eq}}{6}\left( x^{-3} +x^{-3 (1+w)}\right)$$


pela normalização $x\equiv \dfrac{a}{a_{eq}}$, podemos encontrar a relação de troca de variavel t para x ao obter a derivada de x em relação a t.

$$\dfrac{\partial x}{\partial t}= \dfrac{\partial}{\partial t}\left( \dfrac{a}{a_{eq}}\right) $$

como $a_{eq}$ é constante
$$\dot{x}= \dfrac{1}{a_{eq}} \dfrac{\partial a}{\partial t}= \dfrac{a H}{a_{eq}} = xH,$$

rescrevendo  $\dfrac{\partial \delta}{\partial t }$ em termos de x, temos

$$\dfrac{\partial \delta}{\partial t } = \dfrac{\partial x}{\partial t } \dfrac{\partial \delta}{\partial x }= xH \dfrac{d \delta}{d x },$$

rescrevendo a constante de Hubble em termos de x, temos
$$H^2 = \dfrac{8\pi G}{6}\varepsilon_{eq}\left( x^{-3} +x^{-3 (1+w)}\right)$$

$$H^2 = \dfrac{8\pi G}{6}\left( 2\varepsilon_{d}x^3\right)\left( x^{-3} +x^{-3 (1+w)}\right)$$

$$H^2 = \dfrac{8\pi G\varepsilon_{d}}{3}\left( 1 +x^{-3w}\right)$$

$$H = \sqrt{\dfrac{8\pi G\varepsilon_{d}}{3}\left( 1 +x^{-3w}\right)}$$

rescrevendo $\dfrac{\partial^2 \delta}{\partial t^2}$ em termos de x ,temos

$$\dfrac{\partial^2 \delta}{\partial t^2} = \dfrac{\partial }{\partial t }\left(xH \dfrac{d \delta}{d x } \right)= \dfrac{\partial (xH) }{\partial t }\dfrac{d \delta}{d x } + xH\dfrac{\partial }{\partial t }\left( \dfrac{d \delta}{d x } \right)$$

$$\dfrac{\partial^2 \delta}{\partial t^2} =\dfrac{\partial (x) }{\partial t }\dfrac{d(xH) }{dx }\dfrac{d \delta}{d x } + xH\dfrac{\partial x}{\partial t }\dfrac{d}{dx }\left( \dfrac{d \delta}{d x } \right)$$

$$\dfrac{\partial^2 \delta}{\partial t^2} = xH\dfrac{d \delta}{d x }\left( \dfrac{Hd(x) }{dx } + x\dfrac{dH }{dx }\right) + x^2H^2 \dfrac{d^2 \delta}{d x^2 }$$

$$\dfrac{\partial^2 \delta}{\partial t^2} = xH\dfrac{d \delta}{d x }\left( H + x\dfrac{d }{dx }\left( \sqrt{\dfrac{8\pi G\varepsilon_{d}}{3}\left( 1 +x^{-3w}\right)}\right)\right) + x^2H^2 \dfrac{d^2 \delta}{d x^2 }$$

$$\dfrac{\partial^2 \delta}{\partial t^2} = xH\dfrac{d \delta}{d x }\left( H + x\sqrt{\dfrac{8\pi G\varepsilon_{d}}{3}}\dfrac{d }{dx }\left(\sqrt{  1 +x^{-3w}} \right)\right) + x^2H^2 \dfrac{d^2 \delta}{d x^2 }$$

$$\dfrac{\partial^2 \delta}{\partial t^2} = xH\dfrac{d \delta}{d x }\left( H + x\sqrt{\dfrac{8\pi G\varepsilon_{d}}{3}} \left(\dfrac{-3wx^{-3w -1}}{2\sqrt{1 +x^{-3w}}} \right)\right) + x^2H^2 \dfrac{d^2 \delta}{d x^2 }$$

$$\dfrac{\partial^2 \delta}{\partial t^2} = xH\dfrac{d \delta}{d x }\left( H + x\sqrt{\dfrac{8\pi G\varepsilon_{d}(1 +x^{-3w})}{3}} \left(\dfrac{-3wx^{-3w -1}}{2(1 +x^{-3w})} \right)\right) + x^2H^2 \dfrac{d^2 \delta}{d x^2 }$$

$$\dfrac{\partial^2 \delta}{\partial t^2} = xH\dfrac{d \delta}{d x }\left( H + H \left(\dfrac{-3wx^{-3w}}{2(1 +x^{-3w})} \right)\right) + x^2H^2 \dfrac{d^2 \delta}{d x^2 }$$

$$\dfrac{\partial^2 \delta}{\partial t^2} = xH^2\dfrac{d \delta}{d x }+ xH^2 \dfrac{-3wx^{-3w}}{2(1 +x^{-3w})}\dfrac{d \delta}{d x } + x^2H^2 \dfrac{d^2 \delta}{d x^2 }$$

(46) torna-se
$$xH^2\dfrac{d \delta}{d x }+ xH^2 \dfrac{-3wx^{-3w}}{2(1 +x^{-3w})}\dfrac{d \delta}{d x } + x^2H^2 \dfrac{d^2 \delta}{d x^2 } + 2H\left( xH \dfrac{d \delta}{d x }\right)  -\dfrac{24\pi G\varepsilon_{tot}\delta}{6} = 0$$

$$xH^2\dfrac{d \delta}{d x }+ xH^2 \dfrac{-3wx^{-3w}}{2(1 +x^{-3w})}\dfrac{d \delta}{d x } + x^2H^2 \dfrac{d^2 \delta}{d x^2 } + 2xH^2 \dfrac{d \delta}{d x }  -\dfrac{3H^2\delta}{2(1+x^{-3w})} = 0$$

$$x^2(1+x^{-3w}) \dfrac{d^2 \delta}{d x^2 } +x \dfrac{-3wx^{-3w}}{2}\dfrac{d \delta}{d x } + 3x(1+x^{-3w}) \dfrac{d \delta}{d x }  -\dfrac{3}{2}\delta = 0$$

$$x^2(1+x^{-3w}) \dfrac{d^2 \delta}{d x^2 } -\dfrac{3}{2}xwx^{-3w}\dfrac{d \delta}{d x } + 3x(1+x^{-3w}) \dfrac{d \delta}{d x }  -\dfrac{3}{2}\delta = 0$$

$$x^2(1+x^{-3w}) \dfrac{d^2 \delta}{d x^2 }+ 3x(1+(1-\dfrac{w}{2})x^{-3w}) \dfrac{d \delta}{d x }  -\dfrac{3}{2}\delta = 0$$

\begin{equation}
	x^2 (1+x^{-3w})\dfrac{d^2\delta}{dx^2} + \dfrac{3}{2}x(1+ (1-w)x^{-3w})\dfrac{d\delta}{dt} - \dfrac{3}{2}\delta = 0
\end{equation}

Pulamos em (64) o termo proporcional a $c_s^2$ porque ele é determinado por
a pressão da matéria fria sozinha e, portanto, é insignificante. A solução geral
de (64) para um $w = const$ arbitrário é dado por uma combinação linear de funções hipergeométricas. No entanto, pelo menos em dois casos importantes, eles se reduzem a simples funções elementares.

Constante cosmológica w = − 1. Pode-se facilmente verificar que, neste caso,

\begin{equation}
	\delta_1 (x) = \sqrt{1+x^{-3}}
\end{equation}
\textbf {Problema 6.5} Verifique se $\delta_1 (x) $ é uma solução de (64),  \\
Prova:\\
A derivada de primeira ordem de $\delta_1$ em relação há $x$, temos

$$\dfrac{d \delta_1}{dx} = \dfrac{d}{dx}\left( \sqrt{1+x^{-3}}\right)$$

$$\dfrac{d \delta_1}{dx} = - \dfrac{3}{2x^4\sqrt{1+x^{-3}}}$$

A derivada de segunda ordem de $\delta_1$ em relação há $x$, temos

$$\dfrac{d^2 \delta_1}{dx^2} = \dfrac{d}{dx}\left(- \dfrac{3}{2x^4\sqrt{1+x^{-3}}}\right)$$

$$\dfrac{d^2 \delta_1}{dx^2} = -\dfrac{3}{2}\dfrac{d}{dx}\left(- \dfrac{1}{x^4\sqrt{1+x^{-3}}}\right)$$

$$\dfrac{d^2 \delta_1}{dx^2} = -\dfrac{3}{2}\left(\dfrac{-\dfrac{d (x^4\sqrt{1+x^{-3}})}{dx}}{(x^4\sqrt{1+x^{-3}})^2}\right)$$

$$\dfrac{d^2 \delta_1}{dx^2} = \dfrac{3}{2}\left(\dfrac{\sqrt{1+x^{-3}}\dfrac{d (x^4)}{dx}+x^4\dfrac{d(\sqrt{1+x^{-3}})}{dx}}{x^8(1+x^{-3})}\right)$$

$$\dfrac{d^2 \delta_1}{dx^2} = \dfrac{3}{2}\left(\dfrac{4x^3\sqrt{1+x^{-3}}+x^4\left(- \dfrac{3}{2x^4\sqrt{1+x^{-3}}}\right)}{x^8(1+x^{-3})}\right)$$

$$\dfrac{d^2 \delta_1}{dx^2} = \dfrac{6}{x^5\sqrt{1+ x^{-3}} }-\frac{9}{4x^8 \left(1+x^{-3}\right)^{3/2} }$$

$$ \dfrac{d^2 \delta_1}{dx^2} = \dfrac{24x^3 (1+x^{-3}) - 9}{4x^8 \left(1+x^{-3}\right)^{3/2} }$$

$$ \dfrac{d^2 \delta_1}{dx^2} = \dfrac{3 (8x^3 + 5)}{4x^8 \left(1+x^{-3}\right)^{3/2} }$$

$$ \dfrac{d^2 \delta_1}{dx^2} = \dfrac{3 (8x^3 + 5)}{4x^8\dfrac{(1+x^3)}{x^3} \left(1+x^{-3}\right)^{1/2} }$$

$$ \dfrac{d^2 \delta_1}{dx^2} = \dfrac{3 (8x^3 + 5)}{4x^5 (1+x^3)\left(1+x^{-3}\right)^{1/2} }$$

resolvendo cada termo da equação por vez, temos

$$x^2 (1+x^{-3w})\dfrac{d^2\delta_1}{dx^2} = x^2 (1+x^3)\left( \dfrac{3 (8x^3 + 5)}{4x^5 (1+x^3)\left(1+x^{-3}\right)^{1/2} }\right)$$

$$x^2 (1+x^{-3w})\dfrac{d^2\delta_1}{dx^2} = \dfrac{3 (8x^3 + 5)}{4x^3\left(1+x^{-3}\right)^{1/2} }$$

$$\dfrac{3}{2}x(1+ (1-w)x^{-3w})\dfrac{d\delta}{dt} = \dfrac{3}{2}x(1+ 2x^{3})\left(  - \dfrac{3}{2x^4\sqrt{1+x^{-3}}}\right)$$

$$\dfrac{3}{2}x(1+ (1-w)x^{-3w})\dfrac{d\delta}{dt} =- \dfrac{9(1+ 2x^{3})}{4x^3\sqrt{1+x^{-3}}}$$

$$\dfrac{3}{2}\delta = \dfrac{3\sqrt{1+x^{-3}}}{2}$$

substituindo cada termo na equação, obtemos

$$\dfrac{3 (8x^3 + 5)}{4x^3\left(1+x^{-3}\right)^{1/2} } - \dfrac{9(1+ 2x^{3})}{4x^3\sqrt{1+x^{-3}}} - \dfrac{3\sqrt{1+x^{-3}}}{2} = 0$$

$$\dfrac{3 (8x^3 + 5)}{4x^3\left(1+x^{-3}\right)^{1/2} } - \dfrac{9(1+ 2x^{3})}{4x^3\sqrt{1+x^{-3}}} - \dfrac{6x^3(1+x^{-3})}{{4x^3\sqrt{1+x^{-3}}}} = 0$$

$$3 (8x^3 + 5) - 9(1+ 2x^{3}) - 6x^3(1+x^{-3}) =0$$

$$24x^3 + 15 - 9- 18x^{3} - 6x^3 -6 =0$$

$$0 = 0$$

 portanto satisfaz (64). A outra solução pode ser obtida usando as propriedades do
Wronskiano

\end{document}